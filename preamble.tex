%\IEEEoverridecommandlockouts
% The preceding line is only needed to identify funding in the first footnote. If that is unneeded, please comment it out.
% \usepackage{cite}
 \usepackage{amsmath,amsfonts}
% \usepackage{algorithmic}
% \usepackage{graphicx}
 \usepackage{textcomp}
% \usepackage{xcolor}
\usepackage{xspace}
%\usepackage{float}
%\usepackage{balance}
%\usepackage{rotating}
%\usepackage{multirow}
%\usepackage{needspace}
%\usepackage{microtype}
%\usepackage{bold-extra}
%\usepackage{geometry}
%\usepackage{varioref}
%\usepackage{listings}
\usepackage[normalem]{ulem} %emphasize still italic
%\usepackage{ucs}
%\usepackage{times}
\usepackage{url} %to allow line breaks in footnote url
\def\UrlBreaks{\do\/\do-}
%\usepackage{alltt}
%\usepackage{xfrac}
%\usepackage{subfigure}
%\usepackage{stmaryrd}   % for the \shortuparrow
%\usepackage{setspace}
%\usepackage[numbers, sort&compress]{natbib}
%\usepackage{mdwlist}        % support for better spaced lists
%\usepackage{chngpage} % allows for temporary adjustment of side margins
%\usepackage[normalem]{ulem}
\usepackage{subcaption}
\usepackage{tablefootnote}
\usepackage{array, booktabs}
\usepackage{blindtext}
%\PassOptionsToPackage{hyphens}{url}
%\usepackage[pdftex,colorlinks=true,pdfstartview=FitV,linkcolor=black,citecolor=black,urlcolor=black]{hyperref}
%\usepackage{hyperref}
\usepackage{ifthen}
%\usepackage[normalem]{ulem} % for \sout
%\usepackage{todonotes}
\usepackage[shortlabels]{enumitem}
%\usepackage{enumerate}

\usepackage{csvsimple}
\usepackage{supertabular} % for creating tables that can break across multiple pages. It works similarly to the longtable package but is designed to work well in multicolumn environments.
%\usepackage{multicol} %allow the table to break over columns
\usepackage[most]{tcolorbox}
%\usepackage[multiple]{footmisc}
%\usepackage[caption=false]{subfig}
%\usepackage{subfloat}

%\usepackage{arydshln} % for cdashline
%\usepackage{paralist} %to list extends of previous work
%deactivate indentation on new paragraphs
%\usepackage{parskip}
% for wrapping text in table cells
\usepackage{tabularx}
%\usepackage{longtable} %for longtables however it cause problems with two column mode
%--- for dividing references into sections
%\usepackage[
%natbib=true,
%]{biblatex}
%\addbibresource{scg.bib}

\usepackage{fontawesome}      % for icon next to scg URL next to author list on title page
\usepackage{adjustbox} 		% for vertical text in tables
\usepackage{pifont} 		% for crosses and ticks in tables


\newcolumntype{R}[2]{
    >{\adjustbox{angle=#1,lap=\width-(#2),margin*=0.4em 0em 0em 0em}\bgroup}
    l
    <{\egroup}
}
\newcommand*\rot{\multicolumn{1}{R{90}{1em}}} % no optional argument here, please!

% for footnotes 1,2,3..... style
% (traditional approach with \usepackage[multiple]{footmisc} breaks clickable links, overrides behavior)
\let\oldFootnote\footnote
\newcommand\nextToken\relax
\renewcommand\footnote[1]{%
    \oldFootnote{#1}\futurelet\nextToken\isFootnote}
\newcommand\isFootnote{%
    \ifx\footnote\nextToken\textsuperscript{,}\fi}

\renewcommand{\arraystretch}{1.2}

%-------------- background boxes for RQs-------------
\definecolor{gray50}{gray}{.5}
\definecolor{gray40}{gray}{.6}
\definecolor{gray30}{gray}{.7}
\definecolor{gray20}{gray}{.8}
\definecolor{gray10}{gray}{.9}
\definecolor{gray05}{gray}{.95}

\newlength\Linewidth
\def\findlength{\setlength\Linewidth\linewidth
	\addtolength\Linewidth{-4\fboxrule}
	\addtolength\Linewidth{-3\fboxsep}
}
\newenvironment{rqbox}{\par\begingroup
	\setlength{\fboxsep}{5pt}\findlength
	\setbox0=\vbox\bgroup\noindent
	\hsize=0.95\linewidth
	\begin{minipage}{0.95\linewidth}\normalsize}
	{\end{minipage}\egroup
	\textcolor{gray20}{\fboxsep1.5pt\fbox
		{\fboxsep5pt\colorbox{gray05}{\normalcolor\box0}}}
	\endgroup\par\noindent
	\normalcolor\ignorespacesafterend}
\let\Rqbox\rqbox
\let\endRqbox\endrqbox



\newcommand{\rb}[1]{
	
	%\vspace{0.3cm}
	\begin{tcolorbox}[colback=gray!05,%gray background
		colframe=black,% black frame colour
		width=\columnwidth,% Use 5cm total width,
		arc=3mm, auto outer arc,
		boxrule=0.5pt,
		]
		#1
	\end{tcolorbox}
}

\newcounter{Finding}
\stepcounter{Finding}

\newcommand{\roundedbox}[1]{
	\rb{
		\noindent
		\textit{\textbf{Finding \theFinding}. #1}
	}
	\stepcounter{Finding}
}

\newcommand{\needlines}[1]{\Needspace{#1\baselineskip}}
\newcommand{\ra}{$\rightarrow$}
\newcommand{\yellowbox}[1]{\fcolorbox{gray}{yellow}{\bfseries\sffamily\scriptsize#1}}
\newcommand{\triangles}[1]{{\sf\small$\blacktriangleright$\textit{#1}$\blacktriangleleft$}}
\newcommand{\nb}[2]{\nbc{#1}{#2}{orange}}
\newcommand{\here}{\yellowbox{$\Rightarrow$ CONTINUE HERE $\Leftarrow$}}
\newcommand\fix[1]{\nb{FIX}{#1}}

% add more author macros here
\newcommand\tk[1]{\nbc{TK}{#1}{olive}} 
\newcommand\ab[1]{\nbc{AB}{#1}{purple}}
\newcommand\pr[1]{\nbc{PR}{#1}{violet}}
\newcommand\jz[1]{\nbc{JZ}{#1}{brown}}
\newcommand\vk[1]{\nbe{VK}{#1}{red}}
\newcommand\rB[1]{\nbe{Reviewer 2}{#1}{blue}}
\newcommand\rC[1]{\nbe{Reviewer 3}{#1}{magenta}}
\newcommand\ANS[1]{\nbe{Response}{#1}{teal}}
\newcommand\todo[1]{\nb{TO DO}{#1}}


\newcommand\remind[1]{\nbc{REMINDER}{#1}{gray}}
\newcommand\review[1]{\nbc{TO REVIEW}{#1}{orange}}
\newcommand{\ie}{\emph{i.e.},\xspace}
\newcommand{\eg}{\emph{e.g.},\xspace}
\newcommand{\etal}{\emph{et al.}\xspace}
\newcommand{\etc}{\emph{etc.}\xspace}
\newcommand{\github}{{GitHub}\xspace}

\newcommand{\secheaderfield}[1]{\textbf{\code{#1}}}
%\newcommand{\SO}{{Stack Overflow}\xspace}

\newcommand{\darkui}{Dark UI Colors\xspace}
\newcommand{\dynretdel}{Dynamic Retry Delay\xspace}
\newcommand{\avoidextrawork}{Avoid Extraneous Work\xspace}
\newcommand{\raceidle}{Race-to-idle\xspace}
\newcommand{\openwneccesary}{Open Only When Necessary\xspace}
\newcommand{\pushpoll}{Push over Poll\xspace}
\newcommand{\powersave}{Power Save Mode\xspace}
\newcommand{\poweraware}{Power Awareness\xspace}
\newcommand{\reducesize}{Reduce Size\xspace}
\newcommand{\wificell}{WiFi over Cellular\xspace}
\newcommand{\suplog}{Suppress Logs\xspace}
\newcommand{\batchoper}{Batch Operations\xspace}
\newcommand{\cache}{Cache\xspace}
\newcommand{\decreaserate}{Decrease Rate\xspace}
\newcommand{\usrknowbest}{User Knows Best\xspace}
\newcommand{\informusr}{Inform Users\xspace}
\newcommand{\enoresolution}{Enough Resolution\xspace}
\newcommand{\sensorfusion}{Sensor Fusion\xspace}
\newcommand{\killtask}{Kill Abnormal Tasks\xspace}
\newcommand{\noscrinteraction}{No Screen Interaction\xspace}
\newcommand{\avoidgraphic}{Avoid Extraneous Graphics and Animations\xspace}
\newcommand{\mansyncod}{Manual Sync – On Demand\xspace}

\newcommand{\eps}{energy patterns\xspace}
\newcommand{\eaps}{energy antipatterns\xspace}
\newcommand{\EPS}{Energy Patterns\xspace}
\newcommand{\EAPS}{Energy Antipatterns\xspace}
\newcommand{\antipattern}{antipattern\xspace}
\newcommand{\antipatterns}{antipatterns\xspace}
\newcommand{\numberofpattensincode}{15\xspace}

\newcommand{\powergadget}{Intel Power Gadget\xspace}
\newcommand{\rapl}{Running Average Power Limit\xspace}
\newcommand{\drr}{DRD\xspace}
\newcommand{\oown}{OOWN\xspace}

\newcommand{\yes}{\ding{52}}
\newcommand{\no}{\ding{54}}

\newcommand{\smallfootnote}[1]{\footnote{\fontsize{6}{6}\selectfont #1}}
\newcommand{\n}{$\cdot$}
\newcommand{\y}{\checkmark}
\newcommand{\subscript}[1]{$_{\textrm{\footnotesize{#1}}}$}
\newcommand{\superscript}[1]{$^{\textrm{\footnotesize{#1}}}$}
\newcommand{\vertical}[1]{\raisebox{-4em}{\begin{sideways}{#1}\end{sideways}}}
\newcommand{\code}[1]{\texttt{#1}}
\newcommand{\paratitle}[1]{\emph{#1. }}

\newcolumntype{L}[1]{>{\raggedright\arraybackslash}m{#1}} % left aligned column

% given by IEEE template 
\def\BibTeX{{\rm B\kern-.05em{\sc i\kern-.025em b}\kern-.08em
    T\kern-.1667em\lower.7ex\hbox{E}\kern-.125emX}}
    
% ============================================================
% Markup macros for proof-reading
\newboolean{showedits}
\setboolean{showedits}{true} % toggle to show or hide edits
\ifthenelse{\boolean{showedits}}
{
	\newcommand{\meh}[1]{\textcolor{red}{\uwave{#1}}} % please rephrase
	\newcommand{\ins}[1]{\textcolor{blue}{\uline{#1}}} % please insert
	\newcommand{\del}[1]{\textcolor{red}{\sout{#1}}} % please delete
%	\newcommand{\chg}[2]{\textcolor{red}{\sout{#1}}{\ra}\textcolor{blue}{\uline{#2}}} % please change
	\newcommand{\chg}[2]{\textcolor{orange}{\uline{#2}}} % please change
	\newcommand{\nbe}[3]{
		{\colorbox{#3}{\bfseries\sffamily\scriptsize\textcolor{white}{#1}}}
		{\textcolor{#3}{\sf\small$\blacktriangleright$\textit{#2}$\blacktriangleleft$}}}
}{
	\newcommand{\meh}[1]{#1} % please rephrase
	\newcommand{\ins}[1]{#1} % please insert
	\newcommand{\del}[1]{} % please delete
	\newcommand{\chg}[2]{#2}
	\newcommand{\nbe}[3]{}
}

% ============================================================
% Box comments/edits
\ifthenelse{\boolean{showedits}}
{
	\newtcolorbox{inserted}{
		title=Inserted text:,
		colframe=blue,colback=blue!5!white,
		breakable,
		leftrule=0mm, 
		bottomrule=0mm,
		rightrule=0mm,
		toprule=0mm,
		arc=0mm, outer arc=0mm,
		oversize
	}
	\newtcolorbox{deleted}{
		title=Deleted text:,
		colframe=red,colback=red!5!white,
		breakable,
		leftrule=0mm, 
		bottomrule=0mm,
		rightrule=0mm,
		toprule=0mm,
		arc=0mm, outer arc=0mm,
		oversize
	}
	\newtcolorbox{refactored}{
		title=Rewritten text:,
		colframe=blue,colback=red!5!white,
		breakable,
		leftrule=0mm, 
		bottomrule=0mm,
		rightrule=0mm,
		toprule=0mm,
		arc=0mm, outer arc=0mm,
		oversize
	}
}{
	\newenvironment{inserted}{}{}
	\let\deleted\comment
	\newenvironment{refactored}{}{} 
}

% ============================================================
% Put edit comments in a really ugly standout display
\newboolean{showcomments}
\setboolean{showcomments}{true}
%\setboolean{showcomments}{false}
\ifthenelse{\boolean{showcomments}}
{\newcommand{\nbc}[3]{
		{\colorbox{#3}{\bfseries\sffamily\scriptsize\textcolor{white}{#1}}}
		{\textcolor{#3}{\sf\small$\blacktriangleright$\textit{#2}$\blacktriangleleft$}}}
	\newcommand{\version}{\emph{\scriptsize\id}}}
{\newcommand{\nbc}[3]{}
	\newcommand{\version}{}}

% ============================================================
% Constants
\newcounter{qcounter}

% ============================================================================
% Make quotes be italic
\renewenvironment{quote}
{\list{}{\rightmargin\leftmargin}
	\item\relax\begin{it}}
	{\end{it}\endlist}
\newcommand{\ttimes}{\ensuremath{\times}}

% =============================================================================
% Source Code
\definecolor{source}{gray}{0.9}
\lstset{
	language={},
	% characters
	tabsize=3,
	upquote=true,
	escapechar={^},
	keepspaces=true,
	breaklines=true,
	alsoletter={},
	breakautoindent=true,
	columns=fullflexible,
	showstringspaces=false,
	basicstyle=\footnotesize\ttfamily,
	% background
	frame=single,
	framerule=0pt,
	backgroundcolor=\color{source},
	% numbering
	%numbers=left,
	%numbersep=5pt,
	%numberstyle=\tiny,
	%numberfirstline=true,
	% captioning
	captionpos=b,
	numberbychapter=false,
	% formatting (html)
	moredelim=[is][\textbf]{<b>}{</b>},
	moredelim=[is][\textit]{<i>}{</i>},
	moredelim=[is][\uline]{<u>}{</u>}}
\newcommand{\ct}{\lstinline[backgroundcolor=\color{white},basicstyle=\footnotesize\ttfamily]}
\newcommand{\lct}[1]{{\small\tt #1}}

%----------------------------------------------------------------------------
% Custom macros
\newcommand\keyword[1]{[#1]}
% ============================================================
\newboolean{isblinded}
\setboolean{isblinded}{true}
%\setboolean{isblinded}{false}
\ifthenelse{\boolean{isblinded}}
{\newcommand\blind[1]{BLINDED\xspace}}
{\newcommand\blind[1]{#1\xspace}}


%\usepackage[pdftex,colorlinks=true,pdfstartview=FitV,linkcolor=black,citecolor=black,urlcolor=black,bookmarks=false]{hyperref}

%----------------------------------------------------------------------------
%Paper specific
\newcommand{\rqI}{To what extent can we adapt the energy patterns for mobile applications to web applications?}
\newcommand{\rqII}{What are the concerns and challenges of industry developers for energy patterns?}
\newcommand{\rqIII}{How does an energy pattern impact energy consumption?}

\newcommand{\repFolder}[1]{Folder ``\href{}{RP/#1}'' in the Replication package}
\newcommand{\repFile}[1]{File ``\href{}{RP/#1}'' in the Replication package}
%----------------------------------------------------------------------------


